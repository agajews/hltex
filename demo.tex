\documentclass{article}
\begin{equation} line(x) = the end\end{equation}
\newcommand{\calF}{\mathcal{F}}
\newcommand{\calG}{\mathcal{G}}
\newcommand{\calH}{\mathcal{H}}
\newcommand{\calI}{\mathcal{I}}
\newcommand{\calD}{\mathcal{D}}
\newcommand{\calB}{\mathcal{B}}

\title{a Title}
\begin{document}
The document environment is the only one which doesn't need to be indented.
\section{Some Words}
Here are some words that are in this section.
Math is fun, so here's an equation: %hello
\begin{equation}
    f(x) = x^2 + 3
\end{equation}
34


\begin{equation} f(x) = \textbf{one}Liner with inline commands \end{equation}% even in-line comments!

We might want to give our equation a label, like this:
\begin{equation}\label{eq:cubic}
    f(x) = x^3 - 4x^2 + 2
\end{equation}


\begin{equation}
    random me
\end{equation}
you

Or start at the \textbf{middle} of a \begin{equation} line(x) = the end\end{equation}

Or over multiple lines:
\begin{equation}\label{eq:cubic}
    f(x) = x^3 - 4x^2 + 2
    \begin{equation}
        hi

        there
        meeee  % woah
        me
    \end{equation}

    g(x) = f
\end{equation}

% \itemize:
%     \eq:
%         2 nested
%         \pysplice:
%             print("2\n2\n2\n2")
%         \eq:
%             \pysplice:
%                 print("3\n3\n3\n3")
%                 print("me too")
%             \pysplice:
%                 print("3\n3\n3\n3")
%             3
%             \eq:
%                 \eq:
%                     \pysplice:
%                         print("5\n5\n5\n5")
%                     5
%                 4
%             \pysplice:
%                 print("3\n3\n3\n3")
%         2 nested
%     1 nested

We can reference our equation with Equation \ref{eq:cubic}.
This is automatically joined with the non-breaking space \verb{~}.
\end{document}
