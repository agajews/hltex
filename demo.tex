\documentclass{article}
\begin{equation} line(x) = the end\end{equation}
\begin{document}
The document environment is the only one which doesn't need to be indented.
\section{Some Words}
Here are some words that are in this section.
Math is fun, so here's an equation: %hello
\begin{equation}
    f(x) = x^2 + 3
\end{equation}
34

\begin{equation} f(x) = oneLiner\end{equation}

We might want to give our equation a label, like this:
\begin{equation}\label{eq:cubic}
    f(x) = x^3 - 4x^2 + 2
\end{equation}




Or start at the \textbf{middle} of a \begin{equation} line(x) = the end\end{equation}

Or over multiple lines:
\begin{equation}\label{eq:cubic}
    f(x) = x^3 - 4x^2 + 2

    g(x) = f
\end{equation}

We can reference our equation with Equation \ref{eq:cubic}.
This is automatically joined with the non-breaking space \verb{~}.
\end{document}
